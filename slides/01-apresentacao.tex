% -*- coding: utf-8 -*-

\documentclass{beamer}

\usepackage{pri}
% \usepackage{breakurl}

\graphicspath{{./}{figures/}{figures/01-apresentacao/}}

\title{Processamento e Recuperação de Informação}

\subtitle{Apresentação da Disciplina}

\begin{document}

\maketitle

\begin{frame} 
    \frametitle{Apresentação (1)}
    \begin{block}{Professores}
        \begin{itemize}
        \item[] \includegraphics<1>[scale=0.5]{photo-ist14497}
                \includegraphics<2>[scale=0.5]{photo-ist14497-alt} Pável Calado (responsável)
        \item[] \includegraphics<1>[scale=0.5]{photo-ist24686}
                \includegraphics<2>[scale=0.5]{photo-ist24686-alt} Bruno Martins
        \item[] \includegraphics<1>[scale=0.5]{photo-ist170599}
                \includegraphics<2>[scale=0.5]{photo-ist170599-alt} João Monteiro
        \item[] \includegraphics<1>[scale=0.5]{photo-ist191486}
                \includegraphics<2>[scale=0.5]{photo-ist191486-alt} Danielle Vieira
        \end{itemize}
    \end{block}
    \small\hfill (horário de atendimento no Fénix) 
\end{frame}

\begin{frame} 
    \frametitle{Apresentação (2)}
    \begin{block}{Tema da Disciplina}
        Busca, extração e análise de informação expressa textualmente, e.g. existente na World Wide Web.
    \end{block}
    \begin{block}{Aulas}
        \begin{itemize}
        \item Teóricas: Conceitos Fundamentais + Teoria + Exemplos
        \item Laboratório: Problemas Práticos + Exercícios + Apoio ao Projeto
        \end{itemize}
    \end{block}
\end{frame}

\begin{frame} \frametitle{O que vão aprender...}
    \begin{itemize}
	\item Projetar soluções modernas para o processamento, gestão e interrogação de \emph{grandes volumes de informação não estruturada};
	\item \emph{Classificar e agrupar automaticamente} conjuntos de recursos (e.g., grandes conjuntos de documentos de texto) através de características descritivas;
	\item Conceber sistemas para a \emph{recuperação e filtragem da informação} relevante existente em grandes coleções, com base em termos chave, com base em exemplos, ou com base em perfis dos utilizadores;
	\item Conceber sistemas para a \emph{extração de informação} a partir de documentos textuais ou da Web;
	\item \emph{Avaliar comparativamente diferentes sistemas} para a extração, filtragem e recuperação de informação relevante.	
    \end{itemize}
\end{frame}

\begin{frame}
    \frametitle{Material de Apoio}
    \begin{block}{Bibliografia Principal}
        \scriptsize 
        Ricardo Baeza-Yates and Berthier Ribeiro-Neto, \textbf{Modern Information Retrieval}, 2ª ed. (2011)\\
        {\tiny\url{http://www.mir2ed.org}}\\[.5\baselineskip]

        Bing Liu, \textbf{Web Data Mining}, 2ª ed. (2011)\\
        {\tiny\url{http://www.cs.uic.edu/~liub/WebMiningBook.html}}
    \end{block}
    \begin{block}{Bibliografia Secundária}
        \scriptsize 
        Christopher D.  Manning, Prabhakar Raghavan and Hinrich Schütze,
        \textbf{Introduction to Information Retrieval} (2008)\\
        {\tiny\url{http://nlp.stanford.edu/IR-book/}}\\[.5\baselineskip]

        Anand Rajaraman, Jure Leskovec and Jeffrey D. Ullman, \textbf{Mining of Massive Datasets} (2013)\\
        {\tiny\url{http://infolab.stanford.edu/~ullman/mmds.html}}\\[.5\baselineskip]

        Ian H. Witten, Alistair Moffat, Timothy C. Bell, \textbf{Managing Gigabytes: Compressing and Indexing Documents and Images}, 2ª ed. (2000)\\
        {\tiny\url{http://people.eng.unimelb.edu.au/ammoffat/mg/}}
    \end{block}
    Outras referências serão disponibilizadas ao longo do semestre.
\end{frame}

\begin{frame}
    \frametitle{Avaliação}
    \begin{itemize}
    \item Exame (individual) = 70\%; nota mínima = 9.5
        \begin{itemize}
        \item Exame com consulta, mas limitada a \emph{uma folha A4
              manuscrita}.
        \end{itemize}
%    \item Exercícios semanais (grupo) = 10\%
    \item Projeto (em grupo) = 30\%; nota mínima = 9.5
        \begin{itemize}
        \item Entrega de relatório + apresentação final do projeto
        \item Grupos de \emph{3 alunos}.
        \end{itemize}\end{itemize}    
\end{frame}

\begin{frame} 
    \frametitle{Trabalhadores-Estudantes e Época Especial} 
   \begin{block}{Avaliação para Trabalhadores-Estudantes}
   \begin{itemize}
    \item Mesmo método de avaliação;
	\item Alternativamente, alunos podem optar por apenas fazer o exame.
   \end{itemize}
   Quem fizer projeto em grupo será avaliado como aluno regular
   \end{block}
   \begin{block}{Avaliação em Época Especial}
   \begin{itemize}
   \item Avaliação com base num exame.
   \end{itemize}
   \end{block}
\end{frame}

\begin{frame} 
    \frametitle{Datas para Avaliação}
    \begin{itemize}
	\item Projeto: 07/12/2018
        \begin{itemize}
        \item Apresentações na semana de 10/12
        \end{itemize}
    \item Exame 1 : 8/01/2019 - 11h30
    \item Exame 2 : 5/02/2019 - 8h00
    \end{itemize}
\end{frame}


\begin{frame} 
    \frametitle{Programa}
    \begin{enumerate}
    \footnotesize
	\item Introdução à extração e recuperação de informação 
	\item Modelos clássicos de recuperação de informação
    \item Classificação e agrupamento de documentos
	\item Informação não estruturada e extração de informação textual
	\item Avaliação em recuperação e extração de informação 
	\item Análise de hiperligações e recuperação de informação na Web
    \item Aprendizagem automática para ordenação de documentos
    \item Extração de documentos da Web
	\item Pesquisa por similaridade em dados multi-dimensionais 
	\item Implementação de sistemas de recuperação de informação
	\item Aplicações
	\end{enumerate}
\end{frame}

\begin{frame}
    \frametitle{Laboratórios e Implementação}
    \begin{center}
        Linguagem de programação: \textbf{Python} \\[.5\baselineskip]
    \end{center}
    \begin{block}{Recomendações:}
        \begin{itemize}
        \item Comecem a praticar \emph{hoje!}
        \item Formem os grupos \emph{o mais depressa possível}
        \item Usem os vossos portáteis nas aulas de lab., se possível
        \end{itemize}
    \end{block}
\end{frame}

\begin{frame}
    \frametitle{Python}    
    \begin{block}{Para começar:}
        \footnotesize
        \begin{description}
        \item[Python Programming Language] \url{http://www.python.org/}
        \item[The Python Tutorial] \url{http://docs.python.org/tutorial/}
        \item[The Python Standard Library] \url{http://docs.python.org/library/}
        \item[Python Tutorial @ w3schools] \url{https://www.w3schools.com/python/}
        \end{description}
    \end{block}
    \begin{block}{Outras ferramentas úteis:}
        \footnotesize
        \begin{description}
        \item[Natural Language Toolkit] \url{http://nltk.org/}
        \item[Whoosh] \url{http://pypi.python.org/pypi/Whoosh/}
        \item[Beautiful Soup] \url{http://www.crummy.com/software/BeautifulSoup/}
        \item[feedparser] \url{https://github.com/kurtmckee/feedparser}
        \item[NumPy] \url{http://www.numpy.org/}
        \item[scikit-learn] \url{http://scikit-learn.org/}
        \item ... e outras a ser apresentadas ao longo das aulas
        \end{description}
    \end{block}
\end{frame}

\begin{frame}
    \begin{block}{}
        \centering
        \Large
        Mais questões?
    \end{block}
\end{frame}


\end{document}
