\documentclass[svgnames]{beamer}

\usepackage{pri}

\graphicspath{{./}{figures/}{figures/17-similarity-search/}} 

\newcommand{\fdt}{\ensuremath{f_{d,t}}}
\newcommand{\ceil}[1]{\ensuremath{\lceil #1 \rceil}}
\newcommand{\floor}[1]{\ensuremath{\lfloor #1 \rfloor}}
\newcommand{\att}{\ensuremath{\leftarrow}}
\newcommand{\ang}[1]{\ensuremath{\langle #1 \rangle}}

\subtitle{Similarity Search}

\begin{document}

\maketitle
\makeoutline

\begin{frame}
    \frametitle{Bibliography}
    \href{http://www.mmds.org}{Jure Leskovec, Anand Rajaraman, and Jeff Ullman, Mining of Massive Datasets,} Chapter 3
\end{frame}

\begin{frame} \frametitle{High Dimensional Data}

Many real-world problems

Web Search and Text Mining
  Billions of documents, millions of terms
Product Recommendations
  Millions of customers, millions of products
Scene Completion, other graphics problems
  Image features
Online Advertising, Behavioral Analysis
  Customer actions e.g., websites visited, searches
\end{frame}

%%%%%
  
\begin{frame} \frametitle{A Common Metaphor}

Many problems can be expressed as finding \emph{similar} sets:
Find near-neighbors in high-dimensional space
Examples:
  Pages with similar words
    For duplicate detection, classification by topic
  Customers who purchased similar products
    NetFlix users with similar tastes in movies
	Products with similar customer sets
  Images with similar features
  Users who visited the similar websites

\end{frame}

%%%%%
  
\begin{frame} \frametitle{Distance Measures}

We formally define \emph{near neighbors} as points that are a “small distance” apart
For each use case, we need to define what distance means

Two major classes of distance measures:
  A Euclidean distance is based on the locations of points in such a space
  A Non-Euclidean distance is based on properties of points, but not their location in a space
     Cosine similarity, \emph{Jaccard similarity coefficient}
  
\end{frame}

%%%%%
  
\begin{frame} \frametitle{Jaccard Similarity}

The Jaccard Similarity of two sets is the size of their intersection / the size of their union:

$Sim(C_1, C_2) = \frac{|C_1 \cap C_2|}{|C_1 \cup C_2|}$

The Jaccard Distance between sets is 1 minus their Jaccard similarity:

$d(C_1, C_2) = 1 - \frac{|C_1 \cap C_2|}{|C_1 \cup C_2|}$


\includeimage{jaccard}

\end{frame}

%%%%%
  
\section{Finding Similar Items}

%%%%%
  
\begin{frame} \frametitle{Finding Similar Documents}

Goal: Given a large number (N in the millions or billions) of text documents, find pairs that are \emph{near duplicates}

Applications:
  Mirror websites, or approximate mirrors
    Don’t want to show both in a search
  Similar news articles at many news sites
    Cluster articles by “same story”
Problems:
  Many small pieces of one doc can appear out of order in another
  Too many docs to compare all pairs
  Docs are so large or so many that they cannot fit in main memory

\end{frame}

%%%%%
  
\begin{frame} \frametitle{Three Essencial Steps}

1. Shingling: Convert documents, emails, etc., to sets;

2. Minhashing: Convert large sets to short signatures, while preserving similarity;
   Depends on the distance metric

3. Locality-sensitive hashing: Focus on pairs of signatures likely to be from similar documents;

\end{frame}

%%%%%
  
\begin{frame} \frametitle{The Big Picture}

\includeimage{overall}

\end{frame}

%%%%%
  
\section{Shingles}

%%%%%
  
\begin{frame} \frametitle{Documents as High Dimensional Data}

Step 1: Shingling: Convert documents, emails, etc., to sets

Simple approaches...
   Document = set of words appearing in doc
   Document = set of “important” words
...don’t work well for this application.
   Need to account for ordering of words

A different way: \emph{Shingles}

\end{frame}

%%%%%
  
\begin{frame} \frametitle{Shingles}

A k-shingle (or k-gram) for a document is a sequence of k tokens that appears in the doc
  Tokens can be characters, words or something else, depending on application
  Assume tokens = characters for examples
Example: k=2; D1= abcab
  Set of 2-shingles: S(D1)={ab, bc, ca}
  Option: Shingles as a bag (i.e., multi-set), count ab twice

Represent a doc by the set of hash values of its k-shingles

\end{frame}

%%%%%
  
\begin{frame} \frametitle{Compressing Shingles}

To compress long shingles, we can hash them to (say) 4 bytes

Represent a doc by the set of hash values of its k-shingles

Idea: Two documents could (rarely) appear to have shingles in common, when in fact only the hash-values were shared

Example: k=2; D1= abcab
  Set of 2-shingles: S(D1)={ab, bc, ca} 
  Hash the singles: h(D1)={1, 5, 7}

\end{frame}

%%%%%
  
\begin{frame} \frametitle{Similarity Metric for Shingles}


Document D1 = set of k-shingles C1=S(D1)
Equivalently, each document is a 0/1 vector in the space of k-shingles
   Each unique shingle is a dimension
    Vectors are very sparse
A natural similarity measure is the Jaccard similarity:

$Sim(C_1, C_2) = \frac{|C_1 \cap C_2|}{|C_1 \cup C_2|}$

\end{frame}

%%%%%
  
\begin{frame} \frametitle{Working Assumption}

Documents that have lots of shingles in common have similar text, even if the text appears in different order

Careful: You must pick k large enough, or most documents will have most shingles
   k = 5 is OK for short documents
   k = 10 is better for long documents

\end{frame}

%%%%%
  
\begin{frame} \frametitle{Motivation for Minhash/LSH}

Suppose we need to find near-duplicate documents among N=1 million documents

Naïvely, we’d have to compute pairwaise Jaccard similarites for every pair of docs
  i.e, N(N-1)/2 ≈ 5*1011 comparisons
  At 105 secs/day and 106 comparisons/sec, it would take 5 days

For N = 10 million, it takes more than a year...

\end{frame}

\section{Minhashing}

%%%%%
  
\begin{frame} \frametitle{The Big Picture}

\includeimage{overall}

\end{frame}

%%%%%
  
\begin{frame} \frametitle{Encoding Sets as Bit Vectors}

Many similarity problems can be formalized as finding subsets hat have significant intersection
  Encode sets using 0/1 (bit, boolean) vectors
  One dimension per element in the universal set
  Interpret set intersection as bitwise AND, and set union as bitwise OR

Example: C1 = 10111; C2 = 10011
  Size of intersection = 3; size of union = 4, Jaccard similarity (not distance) = 3/4
  d(C1,C2) = 1 – (Jaccard similarity) = 1/4
  
\end{frame}

%%%%%
  
\begin{frame} \frametitle{From Sets to Boolean Matrices}

Rows = elements of the universal set

Columns = sets

1 in row e and column s if and only if e is a member of s

Column similarity is the Jaccard similarity of the sets of their rows with 1

Typical matrix is sparse

\includefigure{matrix}

\end{frame}

%%%%%
  
\begin{frame} \frametitle{Jaccard of Columns}

Each document is a column:
  Example: C1 = 1100011; C2 = 0110010
  Size of intersection = 2; size of union = 5, Jaccard similarity (not distance) = 2/5
  d(C1,C2) = 1 – (Jaccard similarity) = 3/5

We might not really represent the data by a boolean matrix
  Sparse matrices are usually better represented by the list of places where there is a non-zero value
  
\includefigure{matrix}

\end{frame}

%%%%%
  
\begin{frame} \frametitle{Finding Similar Columns}

So far:
  Documents represented as sets of shingles
  Represent sets as boolean vectors in a matrix

Next Goal: Find similar columns
 1) Signatures of columns: small summaries of columns
 2) Examine pairs of signatures to find similar columns 
       Essential that similarities of signatures & columns are related
 3) Optional: check that columns with similar signatures are really similar

Warnings:
  Comparing all pairs may take too much time: job for LSH
  These methods can produce false negatives, and even false positives (if the optional check is not made)
  
\end{frame}

%%%%%
  
\begin{frame} \frametitle{Signatures of Columns}

Key idea: \emph{hash} each column C to a small signature h(C), such that:
  (1) h(C) is small enough that the signature fits in RAM
  (2) sim(C1, C2) is the same as the similarity of signatures h(C1) and h(C2)

Goal: Find a hash function h() such that:
  if sim(C1,C2) is high, then with high prob. $h(C1) \eq h(C2)$
  if sim(C1,C2) is low, then with high prob. $h(C1) \neq h(C2)$

Hash docs into buckets, and expect that most pairs of near duplicate docs hash into the same bucket

\end{frame}

%%%%%
  
\begin{frame} \frametitle{Min-Hashing (1)}

Goal: Find a hash function h() such that:
  if sim(C1,C2) is high, then with high prob. $h(C1) \eq h(C2)$
  if sim(C1,C2) is low, then with high prob. $h(C1) \neq h(C2)$

Clearly, the hash function depends on the similarity metric:
  Not all similarity metrics have a suitable hash function
  There is a suitable hash function for Jaccard similarity: Min-hashing
  
\end{frame}

%%%%%
  
\begin{frame} \frametitle{Min-Hashing (2)}

Imagine the rows of the boolean matrix permuted under random permutation $\pi$
Define a hash function $h_\pi(C)$ = the number of the first (in the permuted order $\pi$) row in which column C has value 1:
  $h_\pi (C) = \min_\pi(C)$
Use several (e.g., 100) independent hash functions (i.e., permutations) to create a signature of a column

\end{frame}

%%%%%
  
\begin{frame} \frametitle{Min-Hashing (3)}

\includefigure{minhash}

\end{frame}

%%%%%
  
\begin{frame} \frametitle{Surprising Property}

Under a random permutation $\pi$, $Pr[hπ(C1) = hπ(C2)] = sim(C1, C2)$

Let X be a set of shingles, X ⊆ [2^{64}], x∈X
Then: Pr[π(y) = min(π(X))] = 1/|X|
  It is equally likely that any y∈X is mapped to the min element
Let x be s.t. π(x) = min(π(C1∪C2))
Then either π(x) = min(π(C1)) if x ∈ C1 , or π(x) = min(π(C2)) if x ∈ C2
So the probability that both are true is the probability x ∈ C1 ∩ C2
Pr[min(π(C1))=min(π(C2))]=|C1∩C2|/|C1∪C2|= sim(C1, C2)

\end{frame}

%%%%%
  
\begin{frame} \frametitle{Similarity for Signatures}

We know Pr[hπ(C1) = hπ(C2)] = sim(C1, C2)
Now generalize to multiple hash functions
The similarity of two signatures is the fraction of the hash functions in which they agree
Note: Because of the minhash property, the similarity of columns is the same as the expected similarity of their signatures

\end{frame}

%%%%%
  
\begin{frame} \frametitle{Example}

\includefigure{minhash}

\end{frame}

%%%%%
  
\begin{frame} \frametitle{Minhash Signatures}

Pick 100 random permutations of the rows
Think of sig(C) as a column vector
Let sig(C)[i] = according to the i-th permutation, the index of the first row that has a 1 in column C
   sig(C)[i] = min (πi(C))
Note: The sketch (signature) of document C is small -- \~100 bytes!
We achieved the goal of \emph{compressing} long bit vectors into short signatures

\end{frame}

%%%%%
  
\section{Locality-sensitive hashing}

%%%%%
  
\begin{frame} \frametitle{The Big Picture}

\includeimage{overall}

\end{frame}

%%%%%
  
\begin{frame} \frametitle{LSH : General Intuition}

Goal: Find documents with Jaccard similarity at least s (for some similarity threshold, e.g., s=0.8)
LSH -- Use a function f(x,y) that tells whether x and y is a candidate pair, i.e. a pair of elements whose similarity must be evaluated
For minhash matrices:
  Hash columns of signature matrix M to many buckets
  Each pair of documents that hashes into the same bucket is a candidate pair
  
\end{frame}

%%%%%
  
\begin{frame} \frametitle{Candidates from Minhash}

Pick a similarity threshold s, a fraction < 1
Columns x and y of M are a candidate pair if their signatures agree on at least fraction s of their rows: 

M (i, x) = M (i, y) for at least fraction s values of i

We expect documents x and y to have the same similarity as their signatures

\end{frame}

%%%%%
  
\begin{frame} \frametitle{LSH for Minhash Signatures}

Big idea: Hash columns of signature matrix M several times
Arrange that (only) similar columns are likely to hash to the same bucket, with high probability
Candidate pairs are those that hash to the same bucket

\end{frame}

%%%%%
  
\begin{frame} \frametitle{Partition M into Bands (1)}

Divide matrix M into b bands of r rows
For each band, hash its portion of each column to a hash table with k buckets
Make k as large as possible
Candidate column pairs are those that hash to the same bucket for ≥ 1 band
Tune b and r to catch most similar pairs, but few non-similar pairs

\end{frame}

%%%%%
  
\begin{frame} \frametitle{Partition M into Bands (2)}

\includefigure{bands}

\end{frame}

%%%%%
  
\begin{frame} \frametitle{Hashing Bands}

\includefigure{hashing-bands}

\end{frame}

%%%%%
  
\begin{frame} \frametitle{Simplifying Assumption}

There are enough buckets that columns are unlikely to hash to the same bucket unless they are identical in a particular band
Hereafter, we assume that \emph{same bucket} means \emph{identical in that band}
Assumption needed only to simplify analysis, not for correctness of algorithm

\end{frame}

%%%%%
  
\begin{frame} \frametitle{Example of Bands (1)}

Assume the following case:
  Suppose 100,000 columns of M (100k docs)
  Signatures of 100 integers (rows)
  Therefore, signatures take 40Mb
  Choose 20 bands of 5 integers/band
Goal: Find pairs of documents that are at least s = 80\% similar

\end{frame}

%%%%%
  
\begin{frame} \frametitle{Example of Bands (2)}

Assume: C1, C2 are 80\% similar
Since s=80% we want C1, C2 to hash to at least one common bucket (at least one band is identical)
Probability C1, C2 identical in one particular band: $(0.8)^5 = 0.328$
Probability C1, C2 are not similar in all of the 20 bands: $(1-0.328)^{20} = 0.00035$
  i.e., about 1/3000th of the 80\%-similar column pairs are false negatives
  We would find 99.965\% pairs of truly similar documents
  
\end{frame}

%%%%%
  
\begin{frame} \frametitle{Example of Bands (3)}

Assume: C1, C2 are 30\% similar
Since s=80% we want C1, C2 to hash to at NO common buckets (all bands should be different)
Probability C1, C2 identical in one particular band: $(0.3)^5 = 0.00243$
Probability C1, C2 identical in at least 1 of 20 bands: $1 - (1 - 0.00243)^{20} = 0.0474$
  In other words, approximately 4.74\% pairs of docs with similarity 30\% end up becoming candidate pairs -- false positives
  
\end{frame}

%%%%%
  
\begin{frame} \frametitle{LSH Involves a Tradeoff}

Pick:
  the number of minhashes (rows of M)
  the number of bands b, and
  the number of rows r per band
to balance false positives/negatives
Example: if we had only 15 bands of 5 rows, the number of false positives would go down, but the number of false negatives would go up

\end{frame}

%%%%%
  
\begin{frame} \frametitle{Analysis of LSH - What We Want}

\includefigure{what-we-want}

\end{frame}

%%%%%
  
\begin{frame} \frametitle{Analysis of LSH - What One Band With One Row Gives}

\includefigure{what-1-band-gives}

\end{frame}

%%%%%
  
\begin{frame} \frametitle{Analysis of LSH - b bands with b rows/band}

Columns C1 and C2 have similarity s
Pick any band (r rows)
  Prob. that all rows in band equal = $s^r$
  Prob. that some row in band unequal = $1 - s^r$
  Prob. that no band identical = $(1 - s^r)^b
  Prob. that at least 1 band identical = $1 - (1 - s^r)^b$
  
\end{frame}

%%%%%
  
\begin{frame} \frametitle{Analysis of LSH - What b Bands With r Rows Gives}

\includefigure{what-1-band-gives}

\end{frame}

%%%%%
  
\begin{frame} \frametitle{False Positives vs. False Negatives}

\includefigure{tradeoff}

\end{frame}

%%%%%
  
\begin{frame} \frametitle{LSH Summary}

Tune to get almost all pairs with similar signatures, but eliminate most pairs that do not have similar signatures
Check in main memory that candidate pairs really do have similar signatures
Optional: In another pass through data, check that the remaining candidate pairs really represent similar documents

\end{frame}

%%%%%
  
\begin{frame} \frametitle{The Big Picture}

\includeimage{overall}

\end{frame}

%%%%%

\finalframe{Questions?}

\end{document}
