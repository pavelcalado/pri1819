\documentclass[svgnames]{beamer}

\usepackage{pri}

\graphicspath{{./}{figures/}{figures/17-similarity-search/}} 

\newcommand{\fdt}{\ensuremath{f_{d,t}}}
\newcommand{\ceil}[1]{\ensuremath{\lceil #1 \rceil}}
\newcommand{\floor}[1]{\ensuremath{\lfloor #1 \rfloor}}
\newcommand{\att}{\ensuremath{\leftarrow}}
\newcommand{\ang}[1]{\ensuremath{\langle #1 \rangle}}

\subtitle{Efficient Similarity Search}

\begin{document}

\maketitle
\makeoutline

\begin{frame}
    \frametitle{Bibliography}
    \href{http://www.mmds.org}{Jure Leskovec, Anand Rajaraman, and Jeff Ullman, Mining of Massive Datasets,} Chapter 3
\end{frame}

\begin{frame} \frametitle{High Dimensional Data}
\begin{block}{Many real-world problems}
\begin{itemize}
\item Web Search and Text Mining
  \begin{itemize}
  \item Billions of documents, millions of terms
  \end{itemize}
\item Product Recommendations
  \begin{itemize}
  \item Millions of customers, millions of products
  \end{itemize}
\item Scene Completion, other graphics problems
  \begin{itemize}
  \item Image features
  \end{itemize}
\item Online Advertising, Behavioral Analysis
  \begin{itemize}
  \item Customer actions (e.g., websites visited, searches)
  \end{itemize}
\end{itemize}
\end{block}
\end{frame}

%%%%%
  
\begin{frame} \frametitle{A Common Metaphor}

Many problems can be expressed as finding \emph{similar} sets.

~\\
Find near-neighbors in high-dimensional space.

\begin{block}{Examples:}
  \begin{itemize}
  \item Pages with similar words
    \begin{itemize}
    \item For duplicate detection, classification by topic
    \end{itemize}
  \item Customers who purchased similar products
    \begin{itemize}
    \item NetFlix users with similar tastes in movies
    \end{itemize}
  \item Products with similar customer sets
  \item Images with similar features
  \item Users who visited the similar websites
  \end{itemize}
\end{block}
\end{frame}

%%%%%
  
\begin{frame} \frametitle{Distance Measures}

We formally define \emph{near neighbors} as points that are a \emph{small distance} apart.

~\\
For each use case, we need to define what distance means.

\begin{block}{Two major classes of distance measures:}
  \begin{itemize}
  \item A Euclidean distance is based on the locations of points in such a space
  \item A Non-Euclidean distance is based on properties of points, but not their location in a space
    \begin{itemize}
    \item Cosine similarity, \emph{Jaccard similarity coefficient}, ...
    \end{itemize}
  \end{itemize}
\end{block}  
\end{frame}

%%%%%
  
\begin{frame} \frametitle{Jaccard Similarity}
The Jaccard Similarity of two sets is the size of their intersection over the size of their union.

\begin{block}{}
\begin{center}
$Sim(C_1, C_2) = \frac{|C_1 \cap C_2|}{|C_1 \cup C_2|}$
\end{center}
\end{block} 

The Jaccard Distance between sets is 1 minus their Jaccard similarity.

\begin{block}{}
\begin{center}
$d(C_1, C_2) = 1 - \frac{|C_1 \cap C_2|}{|C_1 \cup C_2|}$
\end{center}
\end{block} 

\includegraphics[width=10cm]{jaccard}
\end{frame}

%%%%%
  
\section{Finding Similar Items}

%%%%%
  
\begin{frame} \frametitle{Finding Similar Documents}

\begin{block}{Goal:}
Given a large number ($N$ in the millions or billions) of text documents, find pairs that are \emph{near duplicates}
\end{block}

\begin{block}{Applications:}
  \scriptsize
  \begin{itemize}
  \item Mirror websites, or approximate mirrors
    \begin{itemize}\scriptsize
    \item Don’t want to show both in a search
    \end{itemize}
  \item Similar news articles at many news sites
    \begin{itemize}\scriptsize
    \item Cluster articles by \emph{same story}
    \end{itemize}
  \end{itemize}
\end{block}
\begin{block}{Problems:}
  \scriptsize
  \begin{itemize}
  \item Many small pieces of one doc can appear out of order in another
  \item Too many docs to compare all pairs
  \item Docs are so large or so many that they cannot fit in main memory
  \end{itemize}
\end{block}
\end{frame}

%%%%%
  
\begin{frame} \frametitle{Three Essencial Steps}

\begin{enumerate}
\item \emph{Shingling:} Convert documents, emails, etc., to sets;

~\\

\item \emph{Minhashing:} Convert large sets to short signatures, while preserving similarity;
   \begin{itemize}
   \item Depends on the distance metric;
   \end{itemize}

~\\

\item \emph{Locality-sensitive hashing:} Focus on pairs of signatures likely to be from similar documents.
\end{enumerate}

\end{frame}

%%%%%
  
\begin{frame} \frametitle{The Big Picture}

\includegraphics[width=10cm]{overall}

\end{frame}

%%%%%
  
\section{Shingles}

%%%%%
  
\begin{frame} \frametitle{Documents as High Dimensional Data}

\begin{block}{Step 1:}
Shingling: Convert documents, emails, etc., to sets
\end{block}

\begin{itemize}
\item Simple approaches...
   \begin{itemize}
   \item Document = set of words appearing in document
   \item Document = set of {\it important} words
   \end{itemize}
\item ...don’t work well for this application!
   \begin{itemize}
   \item Need to account for ordering of words
   \end{itemize}

\item A different way: \emph{Shingles}
\end{itemize}

\end{frame}

%%%%%
  
\begin{frame} \frametitle{Shingles}

\begin{itemize}
\item A $k$-shingle (or $k$-gram) for a document is a sequence of $k$ tokens that appears in the document
  \begin{itemize}
  \item Tokens can be characters, words or something else, depending on application
  \item Assume tokens = characters for next examples
  \end{itemize}
\end{itemize}

\begin{block}{Example: $k=2$; $D_1=abcab$}
  Set of 2-shingles: $S(D_1)=\{ab, bc, ca\}$
  
  ~\\
  Option: Shingles as a bag (i.e., multi-set), counting $ab$ twice
\end{block}
\begin{itemize}
\item Represent a doc by the set of hash values of its $k$-shingles
\end{itemize}

\end{frame}

%%%%%
  
\begin{frame} \frametitle{Compressing Shingles}

\begin{itemize}
\item To compress long shingles, we can hash them into a convenient/efficient representation (e.g. 4 bytes)

\item Represent a doc by the set of hash values of its $k$-shingles

\item \emph{Idea:} Two documents could (rarely) appear to have shingles in common, when in fact only the hash-values were shared
\end{itemize}

\begin{block}{Example: $k=2$; $D_1=abcab$}
  Set of 2-shingles: $S(D_1)=\{ab, bc, ca\}$
  
  ~\\
  Hash the shingles: $h(D_1)=\{1, 5, 7\}$
\end{block}
\end{frame}

%%%%%
  
\begin{frame} \frametitle{Similarity Metric for Shingles}

\begin{itemize}
\item Document $D_1$ = set of $k$-shingles $C_1=S(D_1)$
\item Equivalently, each document is a 0/1 vector in the space of $k$-shingles
   \begin{itemize}
   \item Each unique shingle is a dimension
   \item Vectors are very sparse
   \end{itemize}
\item A natural similarity measure is the Jaccard similarity:
\end{itemize}

\begin{block}{}
\begin{center}
$Sim(C_1, C_2) = \frac{|C_1 \cap C_2|}{|C_1 \cup C_2|}$
\end{center}
\end{block}
\end{frame}

%%%%%
  
\begin{frame} \frametitle{Working Assumption}

\begin{itemize}
\item Documents that have lots of shingles in common have similar text, even if the text appears in different order

~\\

\item \emph{Careful:} You must pick $k$ large enough, or most documents will have most shingles
   \begin{itemize}
   \item $k = 5$ is OK for short documents
   \item $k = 10$ is better for long documents
   \end{itemize}
\end{itemize}
\end{frame}

%%%%%
  
\begin{frame} \frametitle{Motivation for Minhash/LSH}

\begin{itemize}
\item Suppose we need to find near-duplicate documents among $N=1$ million documents

~\\

\item Naively, we'd have to compute pairwaise Jaccard similarites for every pair of docs
\begin{itemize}
  \item i.e, $\frac{N \times (N-1)}{2} \approx 5 \times 10^{11}$ comparisons

  ~\\
  
  \item At $10^5$ secs/day and $10^6$ comparisons/sec, it would take 5 days to compute all pairwaise Jaccard similarites
\end{itemize}

\item For $N = 10$ million, it takes more than a year...
\end{itemize}
\end{frame}

\section{Minhashing}

%%%%%
  
\begin{frame} \frametitle{The Big Picture}
\includegraphics[width=10cm]{overall}
\end{frame}

%%%%%
  
\begin{frame} \frametitle{Encoding Sets as Bit Vectors}

\begin{block}{}
Many similarity problems can be formalized as finding subsets hat have significant intersection

\begin{itemize}
\item Encode sets using 0/1 (bit, Boolean) vectors
\item One dimension per element in the universal set
\item Interpret set intersection as bitwise AND, and set union as bitwise OR
\end{itemize}
\end{block}

\begin{block}{Example: $C_1 = 10111$; $C_2 = 10011$}
  \begin{itemize}
  \item Size of intersection = 3; 
  \item Size of union = 4
  \item Jaccard similarity (not distance) = 3/4
  \item $d(C_1,C_2) = 1 - ($Jaccard similarity) = 1/4
  \end{itemize}
\end{block}  
\end{frame}

%%%%%
  
\begin{frame} \frametitle{From Sets to Boolean Matrices}

 \begin{columns}[T]
    \begin{column}{0.5\textwidth}
\begin{itemize}
\item Rows = elements of the universal set
\item Columns = sets
\item 1 in row $e$ and column $s$ if and only if $e$ is a member of $s$

~\\

\item Column similarity is the Jaccard similarity of the sets of their rows with 1

~\\

\item Typical matrix is sparse
\end{itemize}
  \end{column}
  \begin{column}{0.5\textwidth}
   \includegraphics[width=5cm]{matrix}
  \end{column}
 \end{columns}
\end{frame}

%%%%%
  
\begin{frame} \frametitle{Jaccard of Columns}

 \begin{columns}[T]
  \begin{column}{0.5\textwidth}
  Each document is a column:

  {\scriptsize
  \begin{block}{Example: $C1 = 1100011; C2 = 0110010$}
  \scriptsize
  \begin{itemize}
  \scriptsize
  \item Size of intersection = 2; 
  \item size of union = 5, 
  \item Jaccard similarity = 2/5
  \item $d(C_1,C_2) = 1 - ($Jaccard similarity) = 3/5
  \end{itemize}
  \end{block}}  

  We might not really represent the data by a Boolean matrix
  \begin{itemize}
  \item Sparse matrices can be represented by the list of places with non-zero values
  \end{itemize}
  \end{column}
  \begin{column}{0.5\textwidth}
   \includegraphics[width=5cm]{matrix}
  \end{column}
 \end{columns}

\end{frame}

%%%%%
  
\begin{frame} \frametitle{Finding Similar Columns}
\begin{block}{So far:}
\begin{itemize}
\item Documents represented as sets of shingles
\item Represent sets as boolean vectors in a matrix
\end{itemize}
\end{block}

\begin{block}{Next Goal: Find similar columns}
 \begin{enumerate}
 \item Signatures of columns: small summaries of columns
 \item Examine pairs of signatures to find similar columns 
       \begin{itemize}
       \item Essential that similarities of signatures and columns are related
       \end{itemize}
 \item \emph{Optional:} check that columns with similar signatures are really similar
 \end{enumerate}
\end{block}

\begin{block}{Warnings:}
\begin{itemize}
  \scriptsize
  \item Comparing all pairs may take too much time: \emph{job for LSH}
  \item These methods can produce false negatives, and even false positives (if the optional check is not made)
\end{itemize}
\end{block}
\end{frame}

%%%%%
  
\begin{frame} \frametitle{Signatures of Columns}

\begin{block}{Key idea:} 
\emph{Hash} each column $C$ to a small signature $h(C)$, such that:
\begin{enumerate}
  \item $h(C)$ is small enough that the signature fits in RAM
  \item $sim(C_1, C_2)$ is the same as the similarity of signatures $h(C_1)$ and $h(C_2)$
\end{enumerate}
\end{block}

\begin{block}{Goal:} 
Find a hash function $h()$ such that:
\begin{itemize}
\item if $sim(C_1,C_2)$ is high, then with high prob. $h(C_1) = h(C_2)$
\item if $sim(C_1,C_2)$ is low, then with high prob. $h(C_1) \neq h(C_2)$
\end{itemize}
\end{block}

Hash docs into buckets, and expect that most pairs of near duplicate documents hash into the same bucket
\end{frame}

%%%%%
  
\begin{frame} \frametitle{Min-Hashing (1)}

\begin{block}{Goal:} 
Find a hash function $h()$ such that:
\begin{itemize}
\item if $sim(C_1,C_2)$ is high, then with high prob. $h(C_1) = h(C_2)$
\item if $sim(C_1,C_2)$ is low, then with high prob. $h(C_1) \neq h(C_2)$
\end{itemize}
\end{block}

Clearly, the hash function depends on the similarity metric:

\begin{itemize}
  \item Not all similarity metrics have a suitable hash function
  \item There is a suitable hash function for Jaccard similarity: \emph{Min-hashing}
\end{itemize}  
\end{frame}

%%%%%
  
\begin{frame} \frametitle{Min-Hashing (2)}

\begin{itemize}
\item Imagine the rows of the boolean matrix permuted under random permutation $\pi$

~\\

\item Define a hash function $h_\pi(C)$ = the number of the first (in the permuted order $\pi$) row in which column $C$ has value 1:
  
  \begin{center}
  $h_\pi (C) = \min_\pi(C)$
  \end{center}

~\\

\item Use several (e.g., 100) independent hash functions (i.e., permutations) to create a signature of a column
\end{itemize}
\end{frame}

%%%%%
  
\begin{frame} \frametitle{Min-Hashing (3)}
\includegraphics[width=10cm]{minhash}
\end{frame}

%%%%%
  
\begin{frame} \frametitle{Surprising Property}

Under a random permutation $\pi$, $Pr[h_\pi(C_1) = h_\pi(C_2)] = sim(C_1, C_2)$

\begin{block}{Sketch of proof:}
\begin{itemize}
\item Let $X$ be a set of shingles, and let $x \in X$
\item Then: $Pr[\pi(y) = \min(\pi(X))] = \frac{1}{|X|}$
\begin{itemize}
\item It is equally likely that any $y \in X$ is mapped to the min element
\end{itemize}
\item Let $x$ be s.t. $\pi(x) = \min(\pi(C_1 \cup C_2))$
\begin{itemize}
\item Then either $\pi(x) = \min(\pi(C_1))$ if $x \in C_1$ , or $\pi(x) = \min(\pi(C_2))$ if $x \in C2$
\item So the probability that both are true is the probability $x \in C_1 \cap C_2$
\end{itemize}
\item $Pr[\min(\pi(C_1))=\min(\pi(C_2))]=\frac{|C1 \cap C2|}{|C1 \cup C2|}= sim(C_1,C_2)$
\end{itemize}
\end{block}{}
\end{frame}

%%%%%
  
\begin{frame} \frametitle{Similarity for Signatures}

\begin{itemize}
\item We know $Pr[h_\pi(C_1) = h_\pi(C_2)] = sim(C_1, C_2)$

~\\

\item Now generalize to multiple hash functions

~\\

\item The similarity of two signatures is the fraction of the hash functions in which they agree

~\\

\item \emph{Note:} Because of the minhash property, the similarity of columns is the same as the expected similarity of their signatures
\end{itemize}
\end{frame}

%%%%%
  
\begin{frame} \frametitle{Example}
\includegraphics[width=10cm]{example}
\end{frame}

%%%%%
  
\begin{frame} \frametitle{Minhash Signatures}

\begin{itemize}
\item Pick 100 random permutations of the rows
\item Think of $sig(C)$ as a column vector
\item Let $sig(C)[i]$ = according to the $i$-th permutation, the index of the first row that has a 1 in column C

\begin{center}   
$sig(C)[i] = \min (\pi_i(C))$
\end{center}

\item \emph{Note:} The sketch (signature) of document $C$ is small -- \~100 bytes!

\item We achieved the goal of \emph{compressing} long bit vectors into short signatures
\end{itemize}
\end{frame}

%%%%%
  
\section{Locality-sensitive hashing}

%%%%%
  
\begin{frame} \frametitle{The Big Picture}
\includegraphics[width=10cm]{overall}
\end{frame}

%%%%%
  
\begin{frame} \frametitle{LSH : General Intuition}

\begin{block}{Goal:}
Find documents with Jaccard similarity at least $s$ (for some similarity threshold, e.g., $s=0.8$)
\end{block}
\begin{block}{Locality-Sensitive Hashing (LSH)}
\begin{itemize}
\item Use a function $f(x,y)$ that tells whether $x$ and $y$ is a candidate pair, i.e. a pair of elements whose similarity must be evaluated
\item For minhash matrices:
  \begin{itemize}
  \item Hash columns of signature matrix $M$ to many buckets
  \item Each pair of documents that hashes into the same bucket is a candidate pair
\end{itemize}
\end{itemize}
\end{block}
\end{frame}

%%%%%
  
\begin{frame} \frametitle{Candidates from Minhash}

\begin{itemize}
\item Pick a similarity threshold $0 < s < 1$
\item Columns $x$ and $y$ of $M$ are a candidate pair if their signatures agree on at least fraction $s$ of their rows: 

\begin{center}
$M(i, x) = M(i, y)$ for at least fraction $s$ values of $i$
\end{center}

\item We expect documents $x$ and $y$ to have the same similarity as their signatures
\end{itemize}
\end{frame}

%%%%%
  
\begin{frame} \frametitle{LSH for Minhash Signatures}

\begin{itemize}
\item \emph{Big idea:} Hash columns of signature matrix $M$ several times

~\\

\item Arrange that (only) similar columns are likely to hash to the same bucket, with high probability

~\\

\item Candidate pairs are those that hash to the same bucket
\end{itemize}
\end{frame}

%%%%%
  
\begin{frame} \frametitle{Partition $M$ into Bands (1)}

\begin{itemize}
\item Divide matrix $M$ into $b$ bands of $r$ rows
\item For each band, hash its portion of each column to a hash table with $k$ buckets
\begin{itemize}
\item Make $k$ as large as possible
\end{itemize}
\item Candidate column pairs are those that hash to the same bucket for 1 band or more
\item Tune $b$ and $r$ to catch most similar pairs, but few non-similar pairs
\end{itemize}
\end{frame}

%%%%%
  
\begin{frame} \frametitle{Partition M into Bands (2)}
\includegraphics[width=10cm]{bands}
\end{frame}

%%%%%
  
\begin{frame} \frametitle{Hashing Bands}
\includegraphics[width=9cm]{hashing-bands}
\end{frame}

%%%%%
  
\begin{frame} \frametitle{Simplifying Assumption}
\begin{itemize}
\item There are enough buckets that columns are unlikely to hash to the same bucket unless they are identical in a particular band

~\\

\item Hereafter, we assume that \emph{same bucket} means \emph{identical in that band}

~\\

\item Assumption needed only to simplify analysis, not for correctness of algorithm
\end{itemize}
\end{frame}

%%%%%
  
\begin{frame} \frametitle{Example of Bands (1)}
\begin{block}{Assume the following case:}
\begin{itemize}
\item Suppose 100,000 columns of M (100k docs)
\item Signatures of 100 integers (rows)
\item Therefore, signatures take 40Mb
\item Choose 20 bands of 5 integers/band
\end{itemize}
\end{block}
\emph{Goal:} Find pairs of documents that are at least s = 80\% similar
\end{frame}

%%%%%
  
\begin{frame} \frametitle{Example of Bands (2)}

\begin{block}{Assume: $C_1, C_2$ are 80\% similar}
\begin{itemize}
\item Since s=80\% we want $C_1, C_2$ to hash to at least one common bucket (at least one band is identical)
\item Probability $C_1, C_2$ identical in one particular band: $(0.8)^5 = 0.328$
\item Probability $C_1, C_2$ are not similar in all of the 20 bands: $(1-0.328)^{20} = 0.00035$
  \begin{itemize}
  \item i.e., about 1/3000th of the 80\%-similar column pairs are false negatives
  \item We would find 99.965\% pairs of truly similar documents
  \end{itemize}
\end{itemize}
\end{block}
\end{frame}

%%%%%
  
\begin{frame} \frametitle{Example of Bands (3)}

\begin{block}{Assume: $C_1, C_2$ are 30\% similar}
\begin{itemize}
\item Since s=80\% we want $C_1, C_2$ to hash to at NO common buckets (all bands should be different)
\item Probability $C_1, C_2$ identical in one particular band: $(0.3)^5 = 0.00243$
\item Probability $C_1, C_2$ identical in at least 1 of 20 bands: $1 - (1 - 0.00243)^{20} = 0.0474$
  \begin{itemize}
  \item In other words, approximately $4.74$\% pairs of docs with similarity 30\% end up becoming candidate pairs -- false positives
  \end{itemize}
\end{itemize}
\end{block}
\end{frame}

%%%%%
  
\begin{frame} \frametitle{LSH Involves a Tradeoff}
\begin{block}{}
Pick:
\begin{itemize}
  \item the number of minhashes (rows of $M$)
  \item the number of bands $b$, and
  \item the number of rows $r$ per band
\end{itemize}
to balance false positives/negatives
\end{block}

\emph{Example:} if we had only 15 bands of 5 rows, the number of false positives would go down, but the number of false negatives would go up
\end{frame}

%%%%%
  
\begin{frame} \frametitle{Analysis of LSH - What we want}
\includegraphics[width=10cm]{what-we-want}
\end{frame}

%%%%%
  
\begin{frame} \frametitle{Analysis of LSH - One band with one row}
\includegraphics[width=10cm]{what-1-band-gives}
\end{frame}

%%%%%
  
\begin{frame} \frametitle{Analysis of LSH - $b$ bands with $b$ rows/band (1)}

Columns $C_1$ and $C_2$ have similarity $s$

~\\

Pick any band ($r$ rows)

\begin{itemize}
  \item Probability that all rows in band equal = $s^r$
  \item Probability that some row in band unequal = $1 - s^r$
  \item Probability that no band identical = $(1 - s^r)^b$
  \item Probability that at least 1 band identical = $1 - (1 - s^r)^b$  
\end{itemize}
\end{frame}

%%%%%
  
\begin{frame} \frametitle{Analysis of LSH - $b$ bands With $r$ rows/band (2)}
\includegraphics[width=10cm]{what-1-band-gives}
\end{frame}

%%%%%
  
\begin{frame} \frametitle{False Positives vs. False Negatives}
\includegraphics[width=10cm]{tradeoff}
\end{frame}

%%%%%
  
\begin{frame} \frametitle{LSH Summary}
\begin{itemize}
\item Tune to get almost all pairs with similar signatures, but eliminate most pairs that do not have similar signatures

~\\
\item Check in main memory that candidate pairs really do have similar signatures

~\\

\item {\it Optional:} In another pass through data, check that the remaining candidate pairs really represent similar documents
\end{itemize}
\end{frame}

%%%%%
  
\begin{frame} \frametitle{The Big Picture}
\includegraphics[width=10cm]{overall}
\end{frame}

%%%%%

\finalframe{Questions?}

\end{document}
