\documentclass[12pt]{article}

\usepackage{prilab}

\begin{document}

\maketitle{Course Project}

\section{Description and Goal}
\label{sec:description}

The goal of this project is to implement an Information Search \& Extraction
System for the analysis of political discourse.

Your system will have access to a large set of documents containing the
electoral manifestos of several political parties from different countries in
the world. Using this data, the system should be able to provide the following
functionalities.

\begin{enumerate}
\item \textit{Ad hoc} search on the collection of documents.

    Given a query, represented by set of keywords, the system should
    \textbf{return all the manifestos containing such keywords}, ordered
    according to their relevance to the query. In addition, the system should
    \textbf{show some statistics}, such as:
    \begin{itemize}
    \item For each party, how many manifestos are in the results returned;
    \item How many times each party mentions each keyword;
    \end{itemize} as any others you may find interesting.

\item Classification of documents according to their political affiliation.

    Given any natural language text, the system should be able to
    \textbf{predict which political party is most likely to have produced such
      text}. In addition, the system should also be able \textbf{test the
      effectiveness of its predictions} by showing the values of precision,
    recall, and F1 that it is able to achieve with the available data.

\item Statistical analysis of the subjects mentioned.

    You document should be able to \textbf{discover all named entities
      mentioned in the manifestos} (e.g. people, companies, countries, etc.)
    and \textbf{provide some statistics on their usage}. These statistics
    should enable us to answer questions such as:
    \begin{itemize}
    \item What are the most mentioned entities for each party?
    \item What are the most mentioned entities globally?
    \item Which party is mentioned more times by the other parties?
    \item How many times does any given party mention other parties?
    \end{itemize} as any others you may find interesting.

\end{enumerate}


\section{Implementation}
\label{sec:implementation}

Your Information Search \& Extraction System should be implemented using the
\textbf{Python programming language}, version~2.7 or version~3.

\textbf{There is no need for a user interface}. You may implement one, if you
wish, but it will not be taken into consideration for grading your work. If
you do implement an interface, you can use any programming language
(e.g. Javascript, etc.).

If you do not implement an interface, all the required functionalities should
be available by \textbf{executing python scripts on the command line}.

Your code should be \textbf{easily readable and properly documented}.

\section{Resources}
\label{sec:resources}

\subsection*{Libraries}
\label{libraries}

To implement the system, you can develop any code necessary and/or use any of
the existing libraries or services available online. Examples of such libraries
are:
\begin{itemize}
\item the \emph{scikit-learn}\footnote{\url{http://scikit-learn.org/}} machine
    learning library;
\item the \emph{NLTK}\footnote{\url{https://www.nltk.org/}} natural language
    processing library;
\item the \emph{Whoosh} search
    engine\footnote{\url{https://whoosh.readthedocs.io/}};
\end{itemize} among others.

\subsection*{Data}
\label{sec:data}

A set of political manifestos to use as data for your project will be
provided. This set has been extracted from the \emph{Manifesto Project}
dataset\footnote{\url{https://manifesto-project.wzb.eu/}} and contains the
textual content of each manifesto, together with additional information, such
as the corresponding political party, the language, the date, etc. You are
free to use this information in any way you see fit.

You can explore the Manifesto Project website, or any other source of
information, to improve your solution. However, \textbf{there is no need to
  download data from the Manifesto Project website}. A pre-processed version of
the documents will be provided on the PRI course page.


\section{Deliverable}
\label{sec:deliverable}

Your project should be delivered in a compressed \emph{zip} file named
\texttt{pri\_project\_group\_XX.zip}, where \texttt{XX} should be replaced by
your group number.

The \emph{zip} file should contain:
\begin{itemize}
\item The project code, within a directory named \texttt{code};
\item A file named \texttt{readme.txt}, containing a succinct set of instructions
    on how to run your code;
\item A \emph{pdf} file named \texttt{pri\_project\_group\_XX\_report.pdf}
    (where \texttt{XX} should be replaced by your group number), containing
    your project report.
\end{itemize}

Your report should be formatted according to the ACM Proceedings template\footnote{Use the \texttt{sigconf} format, available at \url{https://www.acm.org/publications/proceedings-template}.}. The report should have \textbf{at most 6 pages} and should contain:
\begin{itemize}
\item A short abstract (1 paragraph) summarizing the report;
\item A short introduction (1 paragraph) describing the problem\footnote{No philosophy, please --- just state what you are trying to do.};
\item One section describing \textbf{how you implemented your search solution};
\item One section describing \textbf{how you implemented your classification solution};
\item One section describing \textbf{how you implemented your entity analysis solution};
\item One section describing \textbf{your results}. This section should contain:
    \begin{itemize}
    \item For the search solution, examples of successful and unsuccessful
        queries, with an explanation of why that happens;
    \item For the classification solution, a confusion matrix and the corresponding precision, recall, and F1 results, with an explanation of the values found;
    \item For the entity analysis solution, examples of successfully and unsuccessfully extracted entities, with an explanation of why that happens.
    \end{itemize}
\end{itemize}

Describe your solutions \textbf{succinctly, but with enough detail} that they may be
replicated by someone else. When describing the solutions, you should always
\textbf{justify the decisions taken}, i.e. why you chose a certain classifier and not
another, how and why you chose certain parameter values and not others.

You should also \textbf{prepare an oral presentation of the project}, to be
delivered on the last week of classes. The presentation should be set for 10
minutes and include quick explanation and demo of the system
functionalities. Strict timing will be enforced.

\section{Grading}
\label{sec:grading}

The project will be graded according to the following criteria. Each item will
be scored on a scale of [0--5]. The final grade is the sum of all scores,
converted to a [0--20] scale.
\begin{enumerate}
\item \textbf{Originality} (5 = the solution is completely original vs. 0 = only existing libraries where applied, using the default parameters);
\item \textbf{Appropriateness} (5 = each problem has an appropriate solution vs. 0 = the same hack was used everywhere);
\item \textbf{Correctness} (5 = the solution was correctly applied, results were correctly measured, etc. vs. 0 = wrong application of the solution, erroneous results, etc.);
\item \textbf{Exploration} (5 = sufficient effort was made to obtain the best results vs. 0 = the first result obtained was considered good enough);
\item \textbf{Oral presentation} (5 = clear and succinct oral presentation vs. 0 = confusing or incomplete oral presentation)
\item \textbf{Report quality} (5 = good report and code presentation vs. 0 = poorly written report and unreadable code).
\end{enumerate}

All projects must be \textbf{submitted electronically before the deadline}. Delays will
not be accepted.

A \textbf{printed version of the reported should also be delivered} during the first
class reserved for the oral presentations.

\end{document}
