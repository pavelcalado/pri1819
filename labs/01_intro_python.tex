\documentclass[12pt]{article}

\usepackage{prilab}

\begin{document}

\maketitle{Lab 01: Introduction to Python}

\section{}

Implement the \emph{quicksort} algorithm\footnote{\url{https://en.wikipedia.org/wiki/Quicksort}} in Python. Define a function that receives a list of objects and sorts the list in place. If needed, use the following pseudocode as a guide.
\footnotesize
\begin{verbatim}
Quicksort(A as array, low as int, high as int)
    if (low < high)
        pivot_location = Partition(A,low,high)
        Quicksort(A,low, pivot_location - 1)
        Quicksort(A, pivot_location + 1, high)

Partition(A as array, low as int, high as int)
    pivot = A[low]
    leftwall = low
    for i = low + 1 to high
        if (A[i] < pivot) then
            leftwall = leftwall + 1
            swap(A[i], A[leftwall])
    swap(A[low],A[leftwall])
    return (leftwall)
\end{verbatim}
\normalsize

\section{}

Implement a script that reads a list of numeric values from a file (containing one value per line) and prints the same values in ascending order. Use the quicksort function previously defined.

\section{}

Implement a script that reads a text file, containing natural language text, and prints each word it contains and the number of times the word occurs.

\section{}

Implement a script that reads two text files and counts the number of words in common.

\end{document}