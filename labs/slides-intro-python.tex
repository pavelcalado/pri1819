\documentclass{beamer}

\usepackage{pri}
% \usepackage{breakurl}
% \usepackage[T1]{fontenc}
\usepackage{lmodern}

\let\textquotedbl="

\newenvironment{code}{%
  \begin{exampleblock}{}
      \ttfamily
    }{%      
  \end{exampleblock}
}

\newcommand{\triple}{\textquotedbl\textquotedbl\textquotedbl}


\begin{document}

\subtitle{Introduction to Python}

\maketitle

\begin{frame}
    \frametitle{The Python Language}
    \begin{itemize}
    \item Interpreted language
    \item Dynamic typing
    \item Many native high-level data structures (lists, dictionaries, ...)
    \item Many available modules (standard and 3rd party)
    \item (Not strictly) object-oriented 
    \end{itemize}
\end{frame}

\begin{frame}
    \frametitle{The Interpreter}
    \begin{itemize}
    \item The standard interpreter is invoked by:
        \begin{code}
            > python
        \end{code}
    \item For interactive sessions, we recommend using the \emph{ipython}
        toolkit
    \item The ipython shell is invoked by:
        \begin{code}
            > ipython
        \end{code}
    \item Scripts can be run using:
        \begin{code}
            > python scriptname.py
        \end{code}
    \end{itemize}
\end{frame}

\begin{frame}
    \frametitle{Hello world}
    \begin{code}
        > print "Hello world!"
    \end{code}
\end{frame}

\begin{frame}[allowframebreaks]
    \frametitle{Data types}
    \begin{itemize}
    \item Integers (int)
        \begin{code}
            > x = 1\\
            > x = z = 3
        \end{code}
    \item Floating point numbers (float)
        \begin{code}
            > x = 0.45
        \end{code}
    \item Strings (str)
        \begin{code}
            > x = "you can use double quotes"\\
            > x = 'or single quotes: it means the same'\\
            > x = \triple this is a\\
              string with many lines\triple
        \end{code}
    \item Unicode strings (unicode)
        \begin{code}
            > x = u"this is unicode"
        \end{code}
    % \item Complex numbers (complex)
    %     \begin{code}
    %         > x = (1+2j)
    %     \end{code}
    \item To retrieve the type of a variable:
        \begin{code}
            > type(x)
        \end{code}
    \item Conversion between types:
        \begin{code}
            > int('123') \\
            123\\
            > str(345)\\
            '345'
        \end{code}
    \end{itemize}
\end{frame}

\begin{frame}[allowframebreaks]
    \frametitle{String operations}
    \begin{itemize}
    \item Concatenation:
        \begin{code}
            > x = 'abc' + 'def'
        \end{code}
    \item Slicing:
        \begin{code}
            > x = 'abcdef'\\
            > x[3]\\
            'd'\\
            > x[2:4]\\
            'cd'\\
            > x[2:]\\
            'cdef'\\
            > x[-1]\\
            'f'
        \end{code}
    \item String length:
        \begin{code}
            > len(x)
        \end{code}
    \item More string methods:
        \begin{block}{}
            \url{http://docs.python.org/2/library/stdtypes.html\#string-methods}
        \end{block}
    \item More on \emph{unicode strings}:
        \begin{block}{}
            \url{http://farmdev.com/talks/unicode/}
        \end{block}
    \end{itemize}
\end{frame}

\begin{frame}
    \frametitle{Arithmetic operations and comparisons}
    \begin{code}
        > 1 + 2\\
        3\\
        > 4 * 8\\
        32\\
        > 8 / 3\\
        2\\
        > 8 / 3.0\\
        2.6666666666666665\\
        > 3>4\\
        False\\
        > 'ab' != 'ac'\\
        True\\
        > 'yes' if 1>2 else 'no'\\
        'no'
    \end{code}
\end{frame}

\begin{frame}[allowframebreaks]
    \frametitle{Lists}
    \begin{itemize}
    \item A list:
        \begin{code}
            > x = ['abc', 123, 0.5]
        \end{code}
    \item Slicing:
        \begin{code}
            > x[2]\\
            0.5\\
            > x[:2]\\
            x = ['abc', 123]
        \end{code}
    \item Concatenation:
        \begin{code}
            > x = ['abc', 123] + [0.5, 5]
        \end{code}
    \item Deletion:
        \begin{code}
            > del x[1:2]
        \end{code}
    \item Nesting:
        \begin{code}
            > x = ['abc', [3, 5, 6], 123, 0.5]
        \end{code}
    \item List comprehensions:
        \begin{code}
            > values = [2, 5, 8, 13]\\
            > squares = [x**2 for x in values]
        \end{code}
    \item Other \emph{sequence types}:
        \begin{block}{}
            \url{http://docs.python.org/2/library/stdtypes.html\#typesseq}
        \end{block}
    \end{itemize}
\end{frame}

\begin{frame}
    \frametitle{Dictionaries}
    \begin{itemize}
    \item A dictionary:
        \begin{code}
            > x = \{'abc':2, 'efg':3\}
        \end{code}
    \item Access:
        \begin{code}
            > x['abc']
        \end{code}
    \item Keys and values:
        \begin{code}
            > x.keys()\\
            \,['abc', 'efg']\\
            > x.values()\\
            \,[2, 3]
        \end{code}
    \item See also \emph{sets}:
        \begin{block}{}
            \url{http://docs.python.org/2/tutorial/datastructures.html\#sets}
        \end{block}
    \end{itemize}
\end{frame}

\begin{frame}[allowframebreaks]
    \frametitle{Control flow}
    \begin{itemize}
    \item While...
        \begin{code}
            x = 0\\
            while x < 10:\\
            ~~~~print~x\\
            ~~~~x += 1\\
            print 'done'
        \end{code}
        \textbf{Notice the \emph{indentation}!}
    \item If...
        \begin{code}
            if x < 0:\\
            ~~~~print 'neg'\\
            elif x > 0:\\
            ~~~~print 'pos'\\
            else:\\
            ~~~~print 'zero'
        \end{code}
    \item For...
        \begin{code}
            x = ['a', 'b', 'c']\\
            for i in x:\\
            ~~~~print i
        \end{code}
        See also the functions \emph{range} and \emph{xrange} and the
        statements \emph{break}, \emph{continue}, and \emph{else}
    \end{itemize}
\end{frame}

\begin{frame}
    \frametitle{Functions}
    \begin{itemize}
    \item Defining functions:
        \begin{code}
            def add\_one(x):\\
            ~~~~return x + 1
        \end{code}
        Note: functions do not need to return a value
    \item Default argument values:
        \begin{code}
            def add\_value(x, v = 1):\\
            ~~~~return x + v
        \end{code}
    \item Lambda forms:
        \begin{code}
            > f = lambda x: x + 1\\
            > f(3)\\
            4
        \end{code}
    \end{itemize}
\end{frame}

\begin{frame}
    \frametitle{Input and Output}
    \begin{itemize}
    \item Reading and writing files:
        \begin{code}
            f = open(filename, 'r')\\
            for line in f:\\
            ~~~~print line\\
            f.close()
        \end{code}
        \begin{code}
            f = open(filename, 'w')\\
            f.write('abcdef')\\
            f.close()
        \end{code}
    \end{itemize}
\end{frame}

\begin{frame}
    \frametitle{String formatting}
    \begin{code}
        '\{0\}, \{1\}, \{2\}'.format('a', 'b', 'c')
    \end{code}
    \begin{code}
        '\{:<30\}'.format('left aligned')
    \end{code}
    \begin{code}
        "int: \{0:d\};  hex: \{0:x\};  oct: \{0:o\};  bin: \{0:b\}".format(42)
    \end{code}
    \begin{code}
        '\{:+f\}; \{:+f\}'.format(3.14, -3.14)
    \end{code}
    \vfill
    See also:
    \begin{block}{}
        \url{https://docs.python.org/2/library/string.html\#string-formatting}
    \end{block}
\end{frame}

\begin{frame}
    \frametitle{Other stuff}
    \begin{itemize}
    \item Comments (\#) and documentation
    \item Regular expressions
    \item Modules
    \item Classes
    \item Exceptions
    \item The Standard Library
    \end{itemize}
    Look it up...
\end{frame}

\begin{frame}
    \frametitle{Useful Links}    
    \begin{description}
    \item[Python Programming Language] \url{http://www.python.org/}
    \item[The Python Standard Library] \url{http://docs.python.org/2/library/}
    \item[The Python Tutorial] \url{http://docs.python.org/2/tutorial/}
    \item[learnpython.org] \url{http://www.learnpython.org/}
    % acho que este est� velho: \item[Dive Into Python] \url{http://www.diveintopython.net/}
    \item[Google Python Style Guide] \url{http://google-styleguide.googlecode.com/svn/trunk/pyguide.html}
    \item[Python Cheat Sheet] \url{http://www.cheatography.com/davechild/cheat-sheets/python/}
    \end{description}
\end{frame}

\end{document}

%%% Local Variables: 
%%% mode: latex
%%% TeX-master: t
%%% End: 
