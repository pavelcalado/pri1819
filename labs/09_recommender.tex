\documentclass[12pt]{article}

\usepackage{prilab}


\begin{document}

\maketitle{Lab 09: Recommender Systems}

To solve the following problems, you will need to download the MovieLens 100k
dataset, available at \url{http://grouplens.org/datasets/movielens/100k/}. This
is a dataset used for movie recommendation.

From the zip file, you can use only the \emph{u1.base} and \emph{u1.test}
files. Each line in this file contain the user id, the movie id, the rating
assigned by the user to the movie, and a timestamp for the date in which the
rating was given. You do not need to use the timestamp in the following
problems.

\section{}

Read the files and create a \emph{user-item ratings matrix} in memory. Notice
the this matrix is very sparse--- you should use an appropriate data structure.

\section{}

Implement a function that takes as input \emph{the user-item matrix} and
\emph{two user ids}, returning the similarity between both users. The
similarity should be computed through Pearson correlation.

\section{}

Implement a function that takes as input \emph{the user-item matrix}, \emph{one
  user id}, and an integer $N$ and returns the set of $N$ users (user ids) most
similar to the given user.

\section{}

Finally, implement a function that takes as input \emph{the user-item matrix},
\emph{one user id} and \emph{one movie id} and return the predicted rating for
the given user and movie.

\section{Pen and Paper Exercise}

Given the following user-item ratings matrix, compute the similarity between
every user and between every item. Use Pearson correlation for the similarity
between users and the Cosine for the similarity between items.

\begin{center}
    \begin{tabular}{|c|c|c|c|c|}\hline
      & i1 & i2 & i3  & i4\\\hline  
      u1  & 2 & 4 & 3 & 3 \\\hline  
      u1  & 1 & 5 & 5 & 3 \\\hline  
      u2  & 1 & 3 & 3 & ? \\\hline  
      u3  & 3 & 2 & 1 & ? \\\hline  
    \end{tabular}
\end{center}

Fill in the blanks in the matrix with ratings predictions. Use all similar
users to make the prediction.

\end{document}

%%% Local Variables: 
%%% mode: latex
%%% TeX-master: t
%%% End: 

