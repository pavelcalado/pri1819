\documentclass[12pt]{article}

\usepackage{prilab}
\usepackage{url}


\begin{document}

\maketitle{Lab 07: Learning to Rank}

The Whoosh search engine provides three different ranking functions: \emph{BM25}, \emph{TF\_IDF} (i.e. cosine similarity) and \emph{Frequency}\footnote{\url{https://whoosh.readthedocs.io/en/latest/api/scoring.html}}. 

The following example code shows how to perform a query using Whoosh with the
TF\_IDF scoring function and obtain the textual similarity score:
\begin{verbatim}
from whoosh.index import open_dir
from whoosh.qparser import *
ix = open_dir("indexdir")
with ix.searcher(weighting=scoring.TF_IDF()) as searcher:
    query = QueryParser("content", ix.schema, group=OrGroup).parse(u"a query")
    results = searcher.search(query, limit=100)
    for i,r in enumerate(results):
        print r, results.score(i)
\end{verbatim}

The goal of this exercise is to create a method for scoring the documents that combines the results from these three functions.

\section{}

Using the Whoosh search engine with the document collection used in previous labs (files \texttt{pri\_cfc.txt} and \texttt{}), implement a script that performs
searches and returns the results ordered by a \emph{linear combination} of the three textual similarities presented above.

The rank combination formula should be:
\begin{displaymath}
    \operatorname{score}(d,q) = \alpha_1\operatorname{bm25}(d,q) + \alpha_2\operatorname{cos}(d,q) + \alpha_3\operatorname{freq}(d,q)
\end{displaymath}
where $d$ is the document, $q$ is the query, $\operatorname{bm25}$ is the score obtained using the BM25 ranking function, $\operatorname{cos}$ is the score obtained using the TF\_IDF ranking function, and $\operatorname{free}$ is the score obtained using the Frequency ranking function.

Adjust the weights $\alpha_1$, $\alpha_2$, and $\alpha_3$ by hand and try to find an improvement in $F_1$ over the results achieved with each individual ranking
function used in isolation.

\section{}


\section{Pen and Paper Exercise}

TBD

\end{document}

%%% Local Variables: 
%%% mode: latex
%%% TeX-master: t
%%% End: 

